%!TEX encoding = UTF-8 Unicode
% ================================================================================
\documentclass[
    fontsize=12pt,
    headings=small,
    parskip=half,           % Ersetzt manuelles Setzen von parskip/parindent.
    bibliography=totoc,
    numbers=noenddot,       % Entfernt den letzten Punkt der Kapitelnummern.
    open=any,               % Kapitel kann auf jeder Seite beginnen.
%   final                   % Entfernt alle todonotes und den Entwurfstempel.
    ]{scrreprt}
% ===================================Praeambel==================================
%!TEX encoding = UTF-8 Unicode
%!TEX root = hinweiseabschlussarbeit.tex

% Kodierung, Sprache, Patches {{{
\usepackage[T1]{fontenc}    % Ausgabekodierung; ermöglicht Akzente und Umlaute
                            %  sowie korrekte Silbentrennung.
\usepackage[utf8]{inputenc} % Erlaubt die direkte Eingabe spezieller Zeichen;
                            %  utf8 muss die Eingabekodierung des Editors sein.
\usepackage[ngerman]{babel} % Deutsche Sprachanpassungen (z.B. Überschriften).
\usepackage{microtype}      % Optimale Randausrichtung und Skalierung.
\usepackage[
    autostyle,
    ]{csquotes}             % Korrekte Anführungszeichen in der Literaturliste.
%\usepackage{fixltx2e}      % Patches fuer LaTeX2e - seit 2015 nicht mehr nötig
\usepackage{scrhack}        % Verhindert Warnungen mit älteren Paketen.
\usepackage[
  newcommands
]{ragged2e}                 % Verbesserte \ragged...Befehle
\PassOptionsToPackage{
  hyphens
}{url}                      % Sorgt für URL-Umbrüche in Fußzeilen u. Literatur
% }}}

% Schriftarten {{{
\usepackage{mathptmx}       % Times; modifies the default serif and math fonts
\usepackage[scaled=.92]{helvet}% modifies the sans serif font
\usepackage{courier}        % modifies the monospace font
% }}}

% Biblatex {{{
\usepackage[
    style=alphabetic,
    backend=biber,
    %backref=true
    ]{biblatex}             % Biblatex mit alphabetischem Style und biber.
\bibliography{literaturliste.bib} % Dateiname der bib-Datei.
\DeclareFieldFormat*{title}{
    \mkbibemph{#1}}         % Make titles italics
% }}}

% Dokument- und Texteinstellungen {{{
\usepackage[
    a4paper,
    margin=2.54cm,
    marginparwidth=2.0cm,
    footskip=1.0cm
    ]{geometry}             % Ersetzt 'a4wide'.
\clubpenalty=10000          % Keine Einzelzeile am Beginn eines Absatzes
                            %  (Schusterjungen).
\widowpenalty=10000         % Keine Einzelzeile am Ende eines Absatzes
\displaywidowpenalty=10000  %  (Hurenkinder).
\usepackage{floatrow}       % Zentriert alle Floats
\usepackage{ifdraft}        % Ermöglicht \ifoptionfinal{true}{false}
\pagestyle{plain}           % keine Kopfzeilen
% \sloppy                    % großzügige Formatierungsweise
\deffootnote{1em}{1em}{
  \thefootnotemark.\ }      % Verbessert Layout mehrzeiliger Fußnoten
\ifdefined\chapterformat
	\renewcommand*{\chapterformat}{% Hübscht Kapitelüberschrift mit senkrechtem 
		\thechapter\enskip%          grauen Balken zwischen Nummer und Text auf
		\textcolor{gray!50}{\rule[-\dp\strutbox]{2pt}{\baselineskip}}\enskip
	}
\fi
%\setkomafont{disposition}{\normalcolor\bfseries} % Aus der KOMA-Skript-Anleitung: „Mit dieser Änderung verzichten Sie darauf, für alle Gliederungsebenen serifenlose Schrift voreinzustellen“

\makeatletter
\AtBeginDocument{%
    \hypersetup{%
        pdftitle = {\@title},
        pdfauthor  = \@author,
    }
}
\makeatother
% }}}

% Weitere Pakete {{{
\usepackage{graphicx}       % Einfügen von Graphiken.
\usepackage{tabu}           % Einfügen von Tabellen.
\usepackage{multirow}       % Tabellenzeilen zusammenfassen.
\usepackage{multicol}       % Tabellenspalten zusammenfassen.
\usepackage{booktabs}       % Schönere Tabellen (\toprule\midrule\bottomrule).
\usepackage[nocut]{thmbox}  % Theorembox bspw. für Angreifermodell.
\usepackage{amsmath}        % Erweiterte Handhabung mathematischer Formeln.
\usepackage{amssymb}        % Erweiterte mathematische Symbole.
\usepackage{rotating}
\usepackage[
    printonlyused
    ]{acronym}              % Abkürzungsverzeichnis
\usepackage[
    colorinlistoftodos,
    textsize=tiny,          % Notizen und TODOs - mit der todonotes.sty von
    \ifoptionfinal{disable}{}%  Benjamin Kellermann ist das Package "changebar"
    ]{todonotes}            %  bereits integriert.
\usepackage[
    breaklinks,
    hidelinks,
    pdfdisplaydoctitle,
    pdfpagemode = {UseOutlines},
    pdfpagelabels,
    ]{hyperref}             % Sprungmarken im PDF. Lädt das URL-Paket.
    \urlstyle{rm}           % Entfernt die Formattierung von URLs.
%\usepackage{breakurl}
%\def\UrlBreaks{\do\/\do-}
\usepackage{listings}       % Spezielle Umgebung für Quelltextformatierung.
    \lstset{                
        language=C,
        breaklines=true,
        breakatwhitespace=true,
        frame=l,            % Linie links: l, doppelt: L
		framerule=2.5pt,    % Dicke der Linie
		rulecolor=\color{gray},% Farbe der Linie
        captionpos=b,
        xleftmargin=6ex,
        tabsize=4,
        numbers=left,
        numberstyle=\ttfamily\footnotesize,
        basicstyle=\ttfamily\footnotesize,
        keywordstyle=\bfseries\color{green!50!black},
        commentstyle=\itshape\color{magenta!90!black},
        identifierstyle=\ttfamily,
        stringstyle=\color{orange!90!black},
        showstringspaces=false,
        }

\usepackage{algorithm}
\usepackage{algpseudocode}
%\usepackage{filecontents}  % Direktes Einfügen von Dateiinhalt. Wird hier für
                            %  die Verwendung einer .bib-Datei in dieser .tex-
                            %  Datei benötigt.
% }}}

% ===================================Dokument===================================

\title{Intrusion detection for OAuth}
\author{Florian Nehmer}
\date{06.01.2023} % Falls ein bestimmtes Datum eingesetzt werden soll, einfach
                    %  diese Zeile aktivieren.

\begin{document}

\begin{titlepage}% {{{
	\includegraphics[width=6.8cm]{./pic/up-uhh-logo-u-2010-u-farbe-u-rgb.pdf}
	\begin{center}\Large
		\vfill
		Masterarbeit
		\vfill
		\makeatletter
		{\Large\textsf{\textbf{\@title}}\par}
		\makeatother
		\vfill
		vorgelegt von
		\par\bigskip
		\makeatletter
		{\@author} \par
		\makeatother
		Matrikelnummer 6417446 \par
		Studiengang Informatik
		\vfill
		MIN-Fakultät \par
		Fachbereich Informatik
		\vfill
		\makeatletter
		eingereicht am {\@date}
		\makeatother
		\vfill
		Betreuer: Pascal Wichmann, M.\,Sc. Informatik \par
		Erstgutachter: Prof. Dr.-Ing. Hannes Federrath \par
		Zweitgutachter: Pascal Wichmann, M.\,Sc. Informatik.
	\end{center}
	\ifoptionfinal{}{
	\begin{tikzpicture}[remember picture, overlay]
		\node[draw, red, font=\ttfamily\bfseries\Large, xshift=30mm, yshift=238mm,
			rotate=340, text centered, text width=6cm, very thick, rounded
			corners=4mm] at (current page.south) {Entwurf vom \today};
	\end{tikzpicture}
	% ====> Delete me
	\begin{tikzpicture}[overlay]
		\node[draw, blue, font=\sffamily\Large, xshift=0mm, yshift=210mm, rotate=0, text centered, rounded corners=1mm] at (current page.south) {Muster des Deckblatts für Abschlussarbeiten};
	\end{tikzpicture}
	% <==== /Delete me
	}
\end{titlepage}% }}}

\chapter*{Aufgabenstellung}
OAuth [RFC6749] is a widely used authentication protocol, which is typically used between multiple actors, such as different organizations. As authentication is at the core of application security, it is specifically essential to prevent attacks on the authentication.

The tasks of this thesis are as follows: Firstly, a systematic literature study should be performed on existing properties and attacks on the OAuth protocol or its implementations. Secondly, the thesis should design protection strategies for the threats that are not sufficiently solved in existing solutions. Two options for this step are (i) the utilization of anomaly-based intrusion detection for OAuth and (ii) specification-based intrusion detection for OAuth. Thirdly, the thesis should evaluate the security of the designed architecture and compare it to other solutions.

\chapter*{Zusammenfassung}

Für die eilige Leserin bzw. den eiligen Leser sollen auf etwa einer halben, maximal einer Seite die wichtigsten Inhalte, Erkenntnisse, Neuerungen bzw. Ergebnisse der Arbeit beschrieben werden.

Durch eine solche Zusammenfassung (im Engl. auch Abstract genannt) am Anfang der Arbeit wird die Arbeit deutlich aufgewertet. Hier sollte vermittelt werden, warum man die Arbeit lesen sollte.

\tableofcontents

\chapter{Introduction}
\label{chap.introduction}
\section{Motivation}
\section{Research Question}
\section{Outline}

\chapter{Fundamental Knowledge}
\label{chap.fundamental_knowledge}

\section{OAuth 2.0 protocol}
The Open Authorization 2.0 protocol nowadays often referred to as the
\emph{OAuth} protocol, is an authorization framework, that allows third-party
applications to gain limited access to resources in a different location on
behalf of the party, that owns these resources. For many users of the Internet,
it is in practice the protocol behind the ``\emph{Sign in with ...}'' button.
The current standard, first defined in 2012 in RFC6749, is already the
successor of the OAuth 1.0 standard, which was officially published in 2010 by
the IETF in RFC5849 \cite{hammer2010rfc}. In the meantime, several extensions
for the protocol were published as standards and technical reports. These
extensions include new functionalities for the protocol e.g. the ability to use
the protocol with devices like smart TVs and printers \cite{denniss2019oauth}
or documents, which describe several security considerations when implementing
the protocol in practice \cite{lodderstedt2020oauth}. As a whole, the OAuth
working group of the Internet Engineering Task Force (IETF) submitted a total
of 30 Request for Comments (RFCs) and 16 active drafts, from which 7 are active
individual drafts. Table X shows a complete list of all OAuth 2.0. related IETF
submissions by the OAuth working group. 

\subsection{Involved parties}
Because the OAuth protocol is very diverse and complex, as it is a
whole authorization framework it makes sense to narrow it down to its core
features. Starting with the involved parties in the protocol. In general there
are four parties involved in the most common OAuth protocol modes:

\begin{itemize} 

    \item Resource owner: The resource owner is the entity that owns or is
        allowed to manage protected resources. The resource owner might grant
        access to these resources. 

    \item Resource server: The resource server is the server, where the
        protected resources are stored. It can accept or decline authorization
        tokens, which it receives from the client. 

    \item Client: The client is an entity, which makes requests to get access
        to the protected resources, on behalf of the resource owner. 
        
    \item Authorization server: The server, that manages access to the
        protected data. It issues access tokens to the client after successful
        authentication of the resource owner. 

\end{itemize}

Depending on the protocol mode, these four parties or in some cases three
parties exchange different messages in variable ways. In general, there are two
different types of messages, which are explained in the next section.

\subsection{Front-channel and Back-channel messages}

Regarding security considerations messages of the OAuth framework can be
categorized into two main categories, \emph{front-channel} and
\emph{back-channel}. As the protocol is mostly used in the application layer
using HTTP and TLS, \emph{front-channel} means that the message is transported
via the \emph{Request-URI} \cite[Sec. 5.1.2]{fielding1999hypertext} e.g. by
using query parameters. \emph{Back-channel} means on the other hand that the
message is transported via the HTTP message body. In other words, back-channel
messages are transported in one TCP connection, between caller and receiver,
whereas front-channel messages use redirects. \cite[p. 338]{belfaik2022single}.

\subsection{Grant Types}
The protocol flow is dependent on the protocol mode. In the case of OAuth the
different protocol modes are called \emph{Grant types}, as the modes differ on
how the authorization is granted to the resource owner via the client.

\subsubsection{Authorization Code Grant}
According to a recent study, the authorization code grant mode of OAuth is the
most used protocol mode for OAuth on the internet
\cite[Table1]{philippaerts2022oauch}. It is offered by more than 90\% of common 
identity providers.

In this mode illustrated in Figure 2.1, a resource owner aiming to access
protected data via a REST API utilizes a user agent, in this case the web
browser, to interact with a client application that executes API requests
through the user agent. The process begins as the client redirects the user
agent to the authorization server through the front channel. The message
contains a client ID, a redirect URL and a state. The client ID is a unique
identifier of the protected resource. The redirect URI is the location the user
agent is redirected to after authentication. The state can hold any string
value and is most commonly used for CSRF tokens. The authorization server now
asks the resource owner for authentication. If the resource owner is
authenticated and is allowed to access the desired protected resources, the
user agent gets redirected back using the redirect URI. This means that the
message is again sent via the front channel and contains now an authorization
code and the state. Using the valid authorization code and the state, the
client proceeds to request an access token via the back channel at the
authorization server. With the acquired access token the client follows with
the request of the desired protected resource at the resource server.

\begin{figure}[ht]
	\sffamily\footnotesize
	\includegraphics[width=0.6\textwidth]{pic/authorization_code_grant.png}
	\unitlength=0.75mm
	\special{em:linewidth 0.4pt}
	\linethickness{0.4pt}
	\caption{Authorization Code Grant flow without any extensions}
	\label{fig:auth_code_grant}
\end{figure}

\subsubsection{Implicit Grant}
The Implicit Grant comes from a time when there were no mechanisms like
Cross-Origin resource sharing implemented in browsers, to share content from
different domains. It is a predecessor of the authorization code flow and works
similarly with the difference of leaving out the exchanging of the
authorization code step. Instead, the access token is sent via the front
channel directly from the authorization server to the client. This leaves open
more attack vectors for example by utilizing the browser history or by
simplifying access token injection \cite{lodderstedt2020oauth}. The Implicit
Grant is officially deprecated, but still has its relevance, as it is still
offered by 37\% of common identity providers \cite{philippaerts2022oauch}.

\subsection{Resource Owner Password Credentials Grant}
This grant type is special in the way that the client is providing its
authentication credentials for the authorization provider to the resource
provider instead. The resource provider than uses the credentials to retrieve
authorization from the authorization provider. This grant type is only feasible
for the scenerio, that the resource provider is trusted completely \cite[Sec.
4.3.]{hardt2012rfc}.
	

\subsubsection{Client Credentials Grant}
The client credentials grant must only be used by confidential clients
interacting with each other. This means the the clients have the ability to
securely store a secret, which is only accessible by themselves. A common
use-case for this scenario would be machine-to-machine interactions. This grant
type is meant for clients to access their own resources as is the case in
micro-service architectures. The client authenticates at the authorization
server with its client secret and receives an access token, to authenticate at
the resource provider. This means that the resource provider does not need to
verify client secrets, but instead only needs the capability to verify access
tokens. This fact is useful for practical reasons, as a resource provider could
reuse the implementation of access tokens for other grant types it is offering.
\cite[Sec. 4.4.]{hardt2012rfc}

\subsubsection{Device Authorization Grant}
Introduced in RFC8628 the device authorization grant is meant to be used for
devices, that lack a user agent like a web browser or do not offer a convenient
way of entering text \cite{denniss2019oauth}. In this grant type the client is
not interacting through a user agent like a web browser anymore, but instead is
using a device authorization endpoint at authorization provider directly to
initiate an authorization request. The client, then instructs the user to open
a webpage on a secondary device to complete the authorization process using a
displayed code for verification of the session. This OAuth flow still requires
the involved devices to use HTTP for communication, which in general is not a
feasible solution for many IoT-devices. A solution for the popular IoT-protocol
CoAP is proposed by Chung et al. 

\subsubsection{Other grant types}
There are several more grant types, which got introduced to the OAuth standard
over time, which are listed below, but are out of scope for this work: 

\begin{itemize}
	\item Refresh Token Grant
	\item JWT Bearer Grant
	\item UMA Grant
	\item SAML 2.0 Bearer Grant
	\item Token Exchange Grant
\end{itemize}

\subsection{OpenID}

\subsection{The Future: OAuth 2.1}

\section{Intrusion Detection}

\subsection{zeek IDS}

\section{Algorithms}

\chapter{Literature Study: Taxonomy of OAuth Vulnerabilities}
\section{OAuth Vulnerabilities}

\subsection{Insufficient Redirect URI Validation \cite{lodderstedt2020oauth} \cite{wang2019make}}
Authorization Servers need to whitelist redirection URLs in order to make sure,
that an attacker cannot craft a hyperlink, which leads to the victim initiating
an OAuth flow and sending the authorization code or token to an
attacker-controlled domain. Some authorization servers may allow the usage of
patterns in order to allow several domains at once. As well as the absence of
any sort of whitelist mechanism even a pattern-matching functionality could
lead to security problems. Among the possibility that a user is entering
patterns that are too broad and allow the usage of unintended redirect URLs,
the attack surface includes issues with the URL parsing implemented by the
authorization server as shown by Wang et al. \cite{wang2019make}. They
presented several techniques to trick the parser into accepting unintended
domain names, like using squared brackets for IPv6 parsing or the \emph{Evil
Slash Trick}, where the parser does not treat a forward slash as a path
separator, while modern browsers do. Depending on the OAuth grant type in use
this vulnerability leads to different possibilities to exploit it.


\subsubsection{Authorization Code Flow}
\begin{itemize}

    \item The attacker uses techniques like phishing to make its victim open an
        attacker-controlled webpage, which initiates an OAuth flow with the
        vulnerable authorization server.
	
    \item The request is crafted with a valid client ID (which is public
        information), ``code'' as response type and a malicious redirect URI,
        which leads to an attacker-controlled server again.
	
    \item If the user logs in at the authorization server, the authorization
        code now gets transmitted to the attacker's webpage, via the redirect
        URI.
	
    \item The attacker page can now use the received authorization code, to
        retrieve a token 

\end{itemize}

\subsubsection{Implicit Flow}
\todo{Maybe write down open redirection with the implicit flow here regarding redirect\_uri check circumvention}


\subsection{Credential Leakage via Referer Headers}
The Referer HTTP header is a potential attack surface. It can be utilized by a
malicious actor to capture query parameters, which are sent via the front
channel, like the state and the authorization code. The authorization code may
be used to redeem an access token before the victim retrieves it and the state
parameter oftentimes includes a CSRF token, which could potentially open up
vulnerabilities in other parts of the application as explained by Fett et al
\cite{fett2016comprehensive}.

\subsubsection{OAuth Client}
If a client renders third-party content, like advertisements in iframes or
images, before redeeming the authorization code for an access token an
attacker, who places these advertisements or images can capture the code via
the referer header and redeem it for an access token.

\subsubsection{Authorization Server}
At the authorization server, the state parameter could be leaked via the
Referer header, when 3rd party images or advertisements are being rendered on
the page. This may be an issue when the state contains a CSRF token as
explained by Fett at al \cite{fett2016comprehensive}.


\subsection{Credential Leakage via Browser History}
\todo{Find studies about risks of confidential data in browser history} 

\subsection{Mix-Up Attacks}

\subsection{Authorization Code Injection \cite{philippaerts2022oauch}} 
The precondition for an authorization code injection is that an attacker has
successfully stolen an authorization code. This can be accomplished in various
ways for example by tricking the user into installing a malicious browser
extension, using other vulnerabilities in a web app like open redirections, or
abusing proxy auto-configuration files \cite*{philippaerts2022oauch}.

In the case, that the client is using the authorization code flow the attacker
can use the stolen authorization code to fetch an access token before the
client does.

\subsection{Countermeasures}
There are several ways to mitigate the risk of authorization code injection.
For example, if an authorization code got used twice in the case it got stolen
and the attacker, as well as the client, tried to redeem a token, every token
that was redeemed with this authorization code should get invalidated. Another
method is the usage of PKCE to make sure the OAuth flow is only used, between
the two legitimate parties.



\subsection{Access Token Injection}
This kind of attack describes the process of an attacker using a stolen access
token in a legitimate authentication flow, to impersonate the client. If the
implicit flow is available, the attacker can now start a new flow and simply
replace the access token in the authorization servers' response. This will
circumvent any CSRF protection, as there is no difference to a non-compromised
flow. \cite{lodderstedt2020oauth}


\subsection{Cross Site Request Forgery}
This type of attack, often referred to by its abbreviation ``CSRF"", is about
the attacker executing a request in the name of the user, by tricking the user
into executing requests for the attacker including all required authentication,
or authorization information. The default OAuth protocol does not include
protection mechanisms against this type of attack. 

\subsection{PKCE Downgrade Attacks}
If an authorization server is not implemented to require PKCE for all its
flows, it is susceptible to being vulnerable to PKCE downgrade attacks
\cite{philippaerts2022oauch}. Even if it is documented otherwise, attackers
might try to omit PKCE parameters, as the current OAuth 2.0 standard does not
require the usage of the PKCE extension \cite{hardt2012rfc}. 


\subsection{Access Token Leakage at the Ressource Server}

\subsection{Misuse of Stolen Access Tokens}

\subsection{Open Redirection}

\subsection{307 Redirect}

\subsection{TLS Terminating Reverse Proxies}

\subsection{Refresh Token Protection}

\subsection{Client Impersonating Resource Owner}

\subsection{Clickjacking}

\subsection{Authorization Server Redirecting to Phishing Site}

\subsection{Attacks on In-Browser Communication Flows}

\section{Countermeasures}
\subsection{PKCE system}
As an extension to OAuth defined by RFC 7636, ```Proof Key for Code Exchange''
is a technique to mitigate Authorization Code Injection misuse or CSRF attacks
\cite{bradley2015rfc}. Specifically, it extends the authorization code flow 

\chapter{Intrusion Detection: State of the Art}

\chapter{Algorithmic Approach}

\chapter{Experimental Analysis}

\chapter{Conclusion}


% =============================Literaturverzeichnis=============================
\begin{raggedright}         % Schaltet Blocksatz ab, erzeugt ein stimmigeres
                            %  Schriftbild im Literaturverzeichnis.
  \printbibliography        % Falls Biblatex verwendet wird.
  \label{sec:literaturverzeichnis}
\end{raggedright}


% ===================================Anhang=====================================
\appendix
\setcounter{figure}{0}
\renewcommand\thetable{A.\arabic{figure}}
\setcounter{table}{0}
\renewcommand\thetable{A.\arabic{table}}
% ===========================Selbstständigkeitserklärung======================
\chapter*{Eidesstattliche Versicherung} % war: Selbständigkeitserklärung
\vspace{1cm}

\todo[noline]{Bitte verwenden Sie hier in jedem Fall die offizielle von der Prüfungsbehörde vorgegebene Formulierung der Selbständigkeitserklärung.}
%
Hiermit versichere ich an Eides statt, dass ich die vorliegende Arbeit
selbstständig verfasst und keine anderen als die angegebenen Hilfsmittel –
insbesondere keine im Quellenverzeichnis nicht benannten Internet-Quellen –
benutzt habe. Alle Stellen, die wörtlich oder sinngemäß aus Veröffentlichungen
entnommen wurden, sind als solche kenntlich gemacht. Ich versichere weiterhin,
dass ich die Arbeit vorher nicht in einem anderen Prüfungsverfahren eingereicht
habe und die eingereichte schriftliche Fassung der auf dem elektronischen
Speichermedium entspricht.

Ggf. streichen: Ich bin damit einverstanden, dass meine Abschlussarbeit in den
Bestand der Fachbereichsbibliothek eingestellt wird.

\makeatletter
Hamburg, den {\@date}
\makeatother

\vspace{2cm}
\rule{6cm}{0.25pt}\\
\makeatletter
{\@author} \par
\makeatother




% ================================Literaturliste-Muster==============================
\newpage
\thispagestyle{empty}
\label{sec:literaturliste}
\par\textbf{\textsf{Thema:}} Privacy Enhancing Technologies zum Schutz von Kommunikationsbeziehungen
\par\textbf{\textsf{Bearbeiter:}} Eva Musterfrau, Heinz Mustermann
\par\textbf{\textsf{Datum:}} \today
\bigskip
% ====> Delete me
\begin{tikzpicture}[overlay]
    \node[draw, blue, font=\sffamily\Large, xshift=70mm, yshift=0mm, rounded corners=1mm]{Muster der Literaturliste};
\end{tikzpicture}
% <==== /Delete me
\par\textbf{\Large\textsf{Literaturliste}}

% ================================Todo list==============================
\listoftodos
% \todototoc

\end{document}
