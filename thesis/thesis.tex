%!TEX encoding = UTF-8 Unicode
% ================================================================================
\documentclass[
    fontsize=12pt,
    headings=small,
    parskip=half,           % Ersetzt manuelles Setzen von parskip/parindent.
    bibliography=totoc,
    numbers=noenddot,       % Entfernt den letzten Punkt der Kapitelnummern.
    open=any,               % Kapitel kann auf jeder Seite beginnen.
%   final                   % Entfernt alle todonotes und den Entwurfstempel.
    ]{scrreprt}
% ===================================Praeambel==================================
%!TEX encoding = UTF-8 Unicode
%!TEX root = hinweiseabschlussarbeit.tex

% Kodierung, Sprache, Patches {{{
\usepackage[T1]{fontenc}    % Ausgabekodierung; ermöglicht Akzente und Umlaute
                            %  sowie korrekte Silbentrennung.
\usepackage[utf8]{inputenc} % Erlaubt die direkte Eingabe spezieller Zeichen;
                            %  utf8 muss die Eingabekodierung des Editors sein.
\usepackage[ngerman]{babel} % Deutsche Sprachanpassungen (z.B. Überschriften).
\usepackage{microtype}      % Optimale Randausrichtung und Skalierung.
\usepackage[
    autostyle,
    ]{csquotes}             % Korrekte Anführungszeichen in der Literaturliste.
%\usepackage{fixltx2e}      % Patches fuer LaTeX2e - seit 2015 nicht mehr nötig
\usepackage{scrhack}        % Verhindert Warnungen mit älteren Paketen.
\usepackage[
  newcommands
]{ragged2e}                 % Verbesserte \ragged...Befehle
\PassOptionsToPackage{
  hyphens
}{url}                      % Sorgt für URL-Umbrüche in Fußzeilen u. Literatur
% }}}

% Schriftarten {{{
\usepackage{mathptmx}       % Times; modifies the default serif and math fonts
\usepackage[scaled=.92]{helvet}% modifies the sans serif font
\usepackage{courier}        % modifies the monospace font
% }}}

% Biblatex {{{
\usepackage[
    style=alphabetic,
    backend=biber,
    %backref=true
    ]{biblatex}             % Biblatex mit alphabetischem Style und biber.
\bibliography{literaturliste.bib} % Dateiname der bib-Datei.
\DeclareFieldFormat*{title}{
    \mkbibemph{#1}}         % Make titles italics
% }}}

% Dokument- und Texteinstellungen {{{
\usepackage[
    a4paper,
    margin=2.54cm,
    marginparwidth=2.0cm,
    footskip=1.0cm
    ]{geometry}             % Ersetzt 'a4wide'.
\clubpenalty=10000          % Keine Einzelzeile am Beginn eines Absatzes
                            %  (Schusterjungen).
\widowpenalty=10000         % Keine Einzelzeile am Ende eines Absatzes
\displaywidowpenalty=10000  %  (Hurenkinder).
\usepackage{floatrow}       % Zentriert alle Floats
\usepackage{ifdraft}        % Ermöglicht \ifoptionfinal{true}{false}
\pagestyle{plain}           % keine Kopfzeilen
% \sloppy                    % großzügige Formatierungsweise
\deffootnote{1em}{1em}{
  \thefootnotemark.\ }      % Verbessert Layout mehrzeiliger Fußnoten
\ifdefined\chapterformat
	\renewcommand*{\chapterformat}{% Hübscht Kapitelüberschrift mit senkrechtem 
		\thechapter\enskip%          grauen Balken zwischen Nummer und Text auf
		\textcolor{gray!50}{\rule[-\dp\strutbox]{2pt}{\baselineskip}}\enskip
	}
\fi
%\setkomafont{disposition}{\normalcolor\bfseries} % Aus der KOMA-Skript-Anleitung: „Mit dieser Änderung verzichten Sie darauf, für alle Gliederungsebenen serifenlose Schrift voreinzustellen“

\makeatletter
\AtBeginDocument{%
    \hypersetup{%
        pdftitle = {\@title},
        pdfauthor  = \@author,
    }
}
\makeatother
% }}}

% Weitere Pakete {{{
\usepackage{graphicx}       % Einfügen von Graphiken.
\usepackage{tabu}           % Einfügen von Tabellen.
\usepackage{multirow}       % Tabellenzeilen zusammenfassen.
\usepackage{multicol}       % Tabellenspalten zusammenfassen.
\usepackage{booktabs}       % Schönere Tabellen (\toprule\midrule\bottomrule).
\usepackage[nocut]{thmbox}  % Theorembox bspw. für Angreifermodell.
\usepackage{amsmath}        % Erweiterte Handhabung mathematischer Formeln.
\usepackage{amssymb}        % Erweiterte mathematische Symbole.
\usepackage{rotating}
\usepackage[
    printonlyused
    ]{acronym}              % Abkürzungsverzeichnis
\usepackage[
    colorinlistoftodos,
    textsize=tiny,          % Notizen und TODOs - mit der todonotes.sty von
    \ifoptionfinal{disable}{}%  Benjamin Kellermann ist das Package "changebar"
    ]{todonotes}            %  bereits integriert.
\usepackage[
    breaklinks,
    hidelinks,
    pdfdisplaydoctitle,
    pdfpagemode = {UseOutlines},
    pdfpagelabels,
    ]{hyperref}             % Sprungmarken im PDF. Lädt das URL-Paket.
    \urlstyle{rm}           % Entfernt die Formattierung von URLs.
%\usepackage{breakurl}
%\def\UrlBreaks{\do\/\do-}
\usepackage{listings}       % Spezielle Umgebung für Quelltextformatierung.
    \lstset{                
        language=C,
        breaklines=true,
        breakatwhitespace=true,
        frame=l,            % Linie links: l, doppelt: L
		framerule=2.5pt,    % Dicke der Linie
		rulecolor=\color{gray},% Farbe der Linie
        captionpos=b,
        xleftmargin=6ex,
        tabsize=4,
        numbers=left,
        numberstyle=\ttfamily\footnotesize,
        basicstyle=\ttfamily\footnotesize,
        keywordstyle=\bfseries\color{green!50!black},
        commentstyle=\itshape\color{magenta!90!black},
        identifierstyle=\ttfamily,
        stringstyle=\color{orange!90!black},
        showstringspaces=false,
        }

\usepackage{algorithm}
\usepackage{algpseudocode}
%\usepackage{filecontents}  % Direktes Einfügen von Dateiinhalt. Wird hier für
                            %  die Verwendung einer .bib-Datei in dieser .tex-
                            %  Datei benötigt.
% }}}

% ===================================Dokument===================================

\title{Intrusion detection for OAuth}
\author{Florian Nehmer}
\date{06.01.2023} % Falls ein bestimmtes Datum eingesetzt werden soll, einfach
                    %  diese Zeile aktivieren.

\begin{document}

\begin{titlepage}% {{{
	\includegraphics[width=6.8cm]{./pic/up-uhh-logo-u-2010-u-farbe-u-rgb.pdf}
	\begin{center}\Large
		\vfill
		Masterarbeit
		\vfill
		\makeatletter
		{\Large\textsf{\textbf{\@title}}\par}
		\makeatother
		\vfill
		vorgelegt von
		\par\bigskip
		\makeatletter
		{\@author} \par
		\makeatother
		Matrikelnummer 6417446 \par
		Studiengang Informatik
		\vfill
		MIN-Fakultät \par
		Fachbereich Informatik
		\vfill
		\makeatletter
		eingereicht am {\@date}
		\makeatother
		\vfill
		Betreuer: Pascal Wichmann, M.\,Sc. Informatik \par
		Erstgutachter: Prof. Dr.-Ing. Hannes Federrath \par
		Zweitgutachter: Pascal Wichmann, M.\,Sc. Informatik.
	\end{center}
	\ifoptionfinal{}{
	\begin{tikzpicture}[remember picture, overlay]
		\node[draw, red, font=\ttfamily\bfseries\Large, xshift=30mm, yshift=238mm,
			rotate=340, text centered, text width=6cm, very thick, rounded
			corners=4mm] at (current page.south) {Entwurf vom \today};
	\end{tikzpicture}
	}
\end{titlepage}% }}}

\chapter*{Research task}
OAuth [RFC6749] is a widely used authentication protocol, which is typically used between multiple actors, such as different organizations. As authentication is at the core of application security, it is specifically essential to prevent attacks on the authentication.

The tasks of this thesis are as follows: Firstly, a systematic literature study should be performed on existing properties and attacks on the OAuth protocol or its implementations. Secondly, the thesis should design protection strategies for the threats that are not sufficiently solved in existing solutions. Two options for this step are (i) the utilization of anomaly-based intrusion detection for OAuth and (ii) specification-based intrusion detection for OAuth. Thirdly, the thesis should evaluate the security of the designed architecture and compare it to other solutions.

\chapter*{Abstract}
OAuth 2.0 is an authorization framework that has rapidly grown in popularity over the last decade. Whenever internet users click a ``Login with...'' button on a webpage, they implicitly engage in the OAuth protocol and allow a webpage to access their sometimes confidential data somewhere stored at another service. Therefore, authorization providers and their consuming clients must provide a robust and secure implementation of the OAuth protocol, as well as network security measures to protect their users' data from attackers. To aid in this responsibility, this thesis presents an overview and classification of common threats and countermeasures when implementing the OAuth 2.0 protocol. This overview shows that flaws in OAuth leading to misuse of the redirection flow of the protocol are among the most impactful threats, and the most sophisticated countermeasures have been established to mitigate them. Based on these observations, this work presents and implements an intrusion detection approach to identify such attacks in a self-generated dataset. The approach utilizes Word2Vec embeddings to encode the HTTP application data of the network traffic into a numerical representation based on their context semantics. These embeddings then get clustered using the k-means and the self-organizing map algorithm to detect anomalies in the network data. The implemented approach resulted in a detection accuracy of 99,3\%, with a precision of 63\% and a yield of 100\%. The results show that the approach generally can sort out the scarce OAuth traffic but has room for improvement in detecting subtle differences inside the OAuth network traffic itself.

\tableofcontents

\chapter{Introduction}
\label{chap:introduction}
\import{chapters/}{01-introduction.tex}

\chapter{Fundamental Knowledge}
\label{chap:fundamental_knowledge}
\import{chapters/}{02-fundamental_knowledge}

\chapter{Related Work}
\label{chap:related_work}
\import{chapters/}{03-related_work.tex}

\chapter{OAuth Threat Landscape}
\label{chap:oauth_security}
\import{chapters/}{04-oauth_threat_landscape.tex}

\chapter{Experimental Analysis}
\label{chap:experimental_analysis}
\import{chapters/}{05-experiments.tex}

\chapter{Conclusion}
\label{chap:conclusion}
\import{chapters/}{06-conclusion.tex}


% =============================Literaturverzeichnis=============================
\begin{raggedright}         % Schaltet Blocksatz ab, erzeugt ein stimmigeres
                            %  Schriftbild im Literaturverzeichnis.
  \printbibliography        % Falls Biblatex verwendet wird.
  \label{sec:literaturverzeichnis}
\end{raggedright}


% ===================================Anhang=====================================
\appendix
\setcounter{figure}{0}
\renewcommand\thetable{A.\arabic{figure}}
\setcounter{table}{0}
\renewcommand\thetable{A.\arabic{table}}
\newpage
		
% ===========================Selbstständigkeitserklärung======================
\chapter*{Eidesstattliche Versicherung} % war: Selbständigkeitserklärung
\vspace{1cm}

\todo[noline]{Bitte verwenden Sie hier in jedem Fall die offizielle von der Prüfungsbehörde vorgegebene Formulierung der Selbständigkeitserklärung.}
%
Hiermit versichere ich an Eides statt, dass ich die vorliegende Arbeit
selbstständig verfasst und keine anderen als die angegebenen Hilfsmittel –
insbesondere keine im Quellenverzeichnis nicht benannten Internet-Quellen –
benutzt habe. Alle Stellen, die wörtlich oder sinngemäß aus Veröffentlichungen
entnommen wurden, sind als solche kenntlich gemacht. Ich versichere weiterhin,
dass ich die Arbeit vorher nicht in einem anderen Prüfungsverfahren eingereicht
habe und die eingereichte schriftliche Fassung der auf dem elektronischen
Speichermedium entspricht.

Ggf. streichen: Ich bin damit einverstanden, dass meine Abschlussarbeit in den
Bestand der Fachbereichsbibliothek eingestellt wird.

\makeatletter
Hamburg, den {\@date}
\makeatother

\vspace{2cm}
\rule{6cm}{0.25pt}\\
\makeatletter
{\@author} \par
\makeatother




% ================================Literaturliste-Muster==============================
% \newpage
% \thispagestyle{empty}
% \label{sec:literaturliste}
% \par\textbf{\textsf{Thema:}} Privacy Enhancing Technologies zum Schutz von Kommunikationsbeziehungen
% \par\textbf{\textsf{Bearbeiter:}} Eva Musterfrau, Heinz Mustermann
% \par\textbf{\textsf{Datum:}} \today
% \bigskip
% \par\textbf{\Large\textsf{Literaturliste}}

% ================================Todo list==============================
\listoftodos
% \todototoc

\end{document}
