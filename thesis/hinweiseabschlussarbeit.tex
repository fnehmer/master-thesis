%!TEX encoding = UTF-8 Unicode
% ================================================================================
\documentclass[
    fontsize=12pt,
    headings=small,
    parskip=half,           % Ersetzt manuelles Setzen von parskip/parindent.
    bibliography=totoc,
    numbers=noenddot,       % Entfernt den letzten Punkt der Kapitelnummern.
    open=any,               % Kapitel kann auf jeder Seite beginnen.
%   final                   % Entfernt alle todonotes und den Entwurfstempel.
    ]{scrreprt}
% ===================================Praeambel==================================
%!TEX encoding = UTF-8 Unicode
%!TEX root = hinweiseabschlussarbeit.tex

% Kodierung, Sprache, Patches {{{
\usepackage[T1]{fontenc}    % Ausgabekodierung; ermöglicht Akzente und Umlaute
                            %  sowie korrekte Silbentrennung.
\usepackage[utf8]{inputenc} % Erlaubt die direkte Eingabe spezieller Zeichen;
                            %  utf8 muss die Eingabekodierung des Editors sein.
\usepackage[ngerman]{babel} % Deutsche Sprachanpassungen (z.B. Überschriften).
\usepackage{microtype}      % Optimale Randausrichtung und Skalierung.
\usepackage[
    autostyle,
    ]{csquotes}             % Korrekte Anführungszeichen in der Literaturliste.
%\usepackage{fixltx2e}      % Patches fuer LaTeX2e - seit 2015 nicht mehr nötig
\usepackage{scrhack}        % Verhindert Warnungen mit älteren Paketen.
\usepackage[
  newcommands
]{ragged2e}                 % Verbesserte \ragged...Befehle
\PassOptionsToPackage{
  hyphens
}{url}                      % Sorgt für URL-Umbrüche in Fußzeilen u. Literatur
% }}}

% Schriftarten {{{
\usepackage{mathptmx}       % Times; modifies the default serif and math fonts
\usepackage[scaled=.92]{helvet}% modifies the sans serif font
\usepackage{courier}        % modifies the monospace font
% }}}

% Biblatex {{{
\usepackage[
    style=alphabetic,
    backend=biber,
    %backref=true
    ]{biblatex}             % Biblatex mit alphabetischem Style und biber.
\bibliography{literaturliste.bib} % Dateiname der bib-Datei.
\DeclareFieldFormat*{title}{
    \mkbibemph{#1}}         % Make titles italics
% }}}

% Dokument- und Texteinstellungen {{{
\usepackage[
    a4paper,
    margin=2.54cm,
    marginparwidth=2.0cm,
    footskip=1.0cm
    ]{geometry}             % Ersetzt 'a4wide'.
\clubpenalty=10000          % Keine Einzelzeile am Beginn eines Absatzes
                            %  (Schusterjungen).
\widowpenalty=10000         % Keine Einzelzeile am Ende eines Absatzes
\displaywidowpenalty=10000  %  (Hurenkinder).
\usepackage{floatrow}       % Zentriert alle Floats
\usepackage{ifdraft}        % Ermöglicht \ifoptionfinal{true}{false}
\pagestyle{plain}           % keine Kopfzeilen
% \sloppy                    % großzügige Formatierungsweise
\deffootnote{1em}{1em}{
  \thefootnotemark.\ }      % Verbessert Layout mehrzeiliger Fußnoten
\ifdefined\chapterformat
	\renewcommand*{\chapterformat}{% Hübscht Kapitelüberschrift mit senkrechtem 
		\thechapter\enskip%          grauen Balken zwischen Nummer und Text auf
		\textcolor{gray!50}{\rule[-\dp\strutbox]{2pt}{\baselineskip}}\enskip
	}
\fi
%\setkomafont{disposition}{\normalcolor\bfseries} % Aus der KOMA-Skript-Anleitung: „Mit dieser Änderung verzichten Sie darauf, für alle Gliederungsebenen serifenlose Schrift voreinzustellen“

\makeatletter
\AtBeginDocument{%
    \hypersetup{%
        pdftitle = {\@title},
        pdfauthor  = \@author,
    }
}
\makeatother
% }}}

% Weitere Pakete {{{
\usepackage{graphicx}       % Einfügen von Graphiken.
\usepackage{tabu}           % Einfügen von Tabellen.
\usepackage{multirow}       % Tabellenzeilen zusammenfassen.
\usepackage{multicol}       % Tabellenspalten zusammenfassen.
\usepackage{booktabs}       % Schönere Tabellen (\toprule\midrule\bottomrule).
\usepackage[nocut]{thmbox}  % Theorembox bspw. für Angreifermodell.
\usepackage{amsmath}        % Erweiterte Handhabung mathematischer Formeln.
\usepackage{amssymb}        % Erweiterte mathematische Symbole.
\usepackage{rotating}
\usepackage[
    printonlyused
    ]{acronym}              % Abkürzungsverzeichnis
\usepackage[
    colorinlistoftodos,
    textsize=tiny,          % Notizen und TODOs - mit der todonotes.sty von
    \ifoptionfinal{disable}{}%  Benjamin Kellermann ist das Package "changebar"
    ]{todonotes}            %  bereits integriert.
\usepackage[
    breaklinks,
    hidelinks,
    pdfdisplaydoctitle,
    pdfpagemode = {UseOutlines},
    pdfpagelabels,
    ]{hyperref}             % Sprungmarken im PDF. Lädt das URL-Paket.
    \urlstyle{rm}           % Entfernt die Formattierung von URLs.
%\usepackage{breakurl}
%\def\UrlBreaks{\do\/\do-}
\usepackage{listings}       % Spezielle Umgebung für Quelltextformatierung.
    \lstset{                
        language=C,
        breaklines=true,
        breakatwhitespace=true,
        frame=l,            % Linie links: l, doppelt: L
		framerule=2.5pt,    % Dicke der Linie
		rulecolor=\color{gray},% Farbe der Linie
        captionpos=b,
        xleftmargin=6ex,
        tabsize=4,
        numbers=left,
        numberstyle=\ttfamily\footnotesize,
        basicstyle=\ttfamily\footnotesize,
        keywordstyle=\bfseries\color{green!50!black},
        commentstyle=\itshape\color{magenta!90!black},
        identifierstyle=\ttfamily,
        stringstyle=\color{orange!90!black},
        showstringspaces=false,
        }

\usepackage{algorithm}
\usepackage{algpseudocode}
%\usepackage{filecontents}  % Direktes Einfügen von Dateiinhalt. Wird hier für
                            %  die Verwendung einer .bib-Datei in dieser .tex-
                            %  Datei benötigt.
% }}}

% ===================================Dokument===================================

\title{Intrusion detection for OAuth}
\author{Florian Nehmer}
\date{06.01.2023} % Falls ein bestimmtes Datum eingesetzt werden soll, einfach
                    %  diese Zeile aktivieren.

\begin{document}

\begin{titlepage}% {{{
	\includegraphics[width=6.8cm]{./pic/up-uhh-logo-u-2010-u-farbe-u-rgb.pdf}
	\begin{center}\Large
		\vfill
		Masterarbeit
		\vfill
		\makeatletter
		{\Large\textsf{\textbf{\@title}}\par}
		\makeatother
		\vfill
		vorgelegt von
		\par\bigskip
		\makeatletter
		{\@author} \par
		\makeatother
		Matrikelnummer 6417446 \par
		Studiengang Informatik
		\vfill
		MIN-Fakultät \par
		Fachbereich Informatik
		\vfill
		\makeatletter
		eingereicht am {\@date}
		\makeatother
		\vfill
		Betreuerin: Pascal Wichmann, M.\,Sc. Informatik \par
		Erstgutachter: Prof. Dr.-Ing. Hannes Federrath \par
		Zweitgutachter: Pascal Wichmann, M.\,Sc. Informatik.
	\end{center}
	\ifoptionfinal{}{
	\begin{tikzpicture}[remember picture, overlay]
		\node[draw, red, font=\ttfamily\bfseries\Large, xshift=30mm, yshift=238mm,
			rotate=340, text centered, text width=6cm, very thick, rounded
			corners=4mm] at (current page.south) {Entwurf vom \today};
	\end{tikzpicture}
	% ====> Delete me
	\begin{tikzpicture}[overlay]
		\node[draw, blue, font=\sffamily\Large, xshift=0mm, yshift=210mm, rotate=0, text centered, rounded corners=1mm] at (current page.south) {Muster des Deckblatts für Abschlussarbeiten};
	\end{tikzpicture}
	% <==== /Delete me
	}
\end{titlepage}% }}}

\chapter*{Aufgabenstellung}
OAuth [RFC6749] is a widely used authentication protocol, which is typically used between multiple actors, such as different organizations. As authentication is at the core of application security, it is specifically essential to prevent attacks on the authentication.

The tasks of this thesis are as follows: Firstly, a systematic literature study should be performed on existing properties and attacks on the OAuth protocol or its implementations. Secondly, the thesis should design protection strategies for the threats that are not sufficiently solved in existing solutions. Two options for this step are (i) the utilization of anomaly-based intrusion detection for OAuth and (ii) specification-based intrusion detection for OAuth. Thirdly, the thesis should evaluate the security of the designed architecture and compare it to other solutions.

\chapter*{Zusammenfassung}

Für die eilige Leserin bzw. den eiligen Leser sollen auf etwa einer halben, maximal einer Seite die wichtigsten Inhalte, Erkenntnisse, Neuerungen bzw. Ergebnisse der Arbeit beschrieben werden.

Durch eine solche Zusammenfassung (im Engl. auch Abstract genannt) am Anfang der Arbeit wird die Arbeit deutlich aufgewertet. Hier sollte vermittelt werden, warum man die Arbeit lesen sollte.

\tableofcontents

\chapter{Introduction}
\label{chap.introduction}

\chapter{Fundamental Knowledge}
\label{chap.fundamental_knowledge}

\section{OAuth}
\subsection{Grant Types}
\begin{itemize}
	\item Authorization Code Grant
	\item Implicit Grant
	\item Resource Owner Password Credentials Grant
	\item Client Credentials Grant
	\item Refresh Token Grant
	\item JWT Bearer Grant
	\item Device Code Grant
	\item UMA Grant
	\item SAML 2.0 Bearer Grant
	\item Token Exchange Grant
\end{itemize}
\todo{Decide on how to handle niche grant types}

\subsection{OpenID}

\subsection{The Future: OAuth 2.1}

\section{Intrusion Detection}

\subsection{zeek IDS}

\section{Algorithms}

\chapter{Taxonomy of OAuth Vulnerabilities}
\section{OAuth Vulnerabilities}

\chapter{State of the Art}

\chapter{Algorithmic Approach}

\chapter{Experimental Analysis}

\chapter{Conclusion}


% =============================Literaturverzeichnis=============================
\begin{raggedright}         % Schaltet Blocksatz ab, erzeugt ein stimmigeres
                            %  Schriftbild im Literaturverzeichnis.
  \printbibliography        % Falls Biblatex verwendet wird.
  \label{sec:literaturverzeichnis}
\end{raggedright}


% ===================================Anhang=====================================
\appendix
\setcounter{figure}{0}
\renewcommand\thetable{A.\arabic{figure}}
\setcounter{table}{0}
\renewcommand\thetable{A.\arabic{table}}
% ===========================Selbstständigkeitserklärung======================
\chapter*{Eidesstattliche Versicherung} % war: Selbständigkeitserklärung
\vspace{1cm}

\todo[noline]{Bitte verwenden Sie hier in jedem Fall die offizielle von der Prüfungsbehörde vorgegebene Formulierung der Selbständigkeitserklärung.}
%
Hiermit versichere ich an Eides statt, dass ich die vorliegende Arbeit selbstständig verfasst und keine anderen als die angegebenen Hilfsmittel – insbesondere keine im Quellenverzeichnis nicht benannten Internet-Quellen – benutzt habe. Alle Stellen, die wörtlich oder sinngemäß aus Veröffentlichungen entnommen wurden, sind als solche kenntlich gemacht. Ich versichere weiterhin, dass ich die Arbeit vorher nicht in einem anderen Prüfungsverfahren eingereicht habe und die eingereichte schriftliche Fassung der auf dem elektronischen Speichermedium entspricht.

Ggf. streichen: Ich bin damit einverstanden, dass meine Abschlussarbeit in den Bestand der Fachbereichsbibliothek eingestellt wird.

\makeatletter
Hamburg, den {\@date}
\makeatother

\vspace{2cm}
\rule{6cm}{0.25pt}\\
\makeatletter
{\@author} \par
\makeatother




% ================================Literaturliste-Muster==============================
\newpage
\thispagestyle{empty}
\label{sec:literaturliste}
\par\textbf{\textsf{Thema:}} Privacy Enhancing Technologies zum Schutz von Kommunikationsbeziehungen
\par\textbf{\textsf{Bearbeiter:}} Eva Musterfrau, Heinz Mustermann
\par\textbf{\textsf{Datum:}} \today
\bigskip
% ====> Delete me
\begin{tikzpicture}[overlay]
    \node[draw, blue, font=\sffamily\Large, xshift=70mm, yshift=0mm, rounded corners=1mm]{Muster der Literaturliste};
\end{tikzpicture}
% <==== /Delete me
\par\textbf{\Large\textsf{Literaturliste}}

% ================================Todo list==============================
\listoftodos
% \todototoc

\end{document}
